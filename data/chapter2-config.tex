% !Mode:: "TeX:UTF-8"
\chapter{环境配置}

\section{C\TeX{}套装 [Windows Only]}

C\TeX{}套装是Windows下为中文优化的\LaTeX{}系统套件,主要基于MiKTeX系统,
集成了编辑器WinEdt和其他相关软件。整个系统封装在一个安装程序中,
安装方法与常规软件相同,无需任何配置,适合大部分Windows用户使用。

\begin{description}
	\item[下载地址] \hfill
	\begin{description}
		\item[官方页面]
		\url{http://www.ctex.org/CTeXDownload}
		\item[清华镜像]
		\url{https://mirrors.tuna.tsinghua.edu.cn/ctex/unstable/}
		\item[中科大镜像]
		\url{http://mirrors.ustc.edu.cn/ctex/unstable/}
	\end{description}
	\item[安装方法] \hfill
	\begin{itemize}
		\item[] 与常规软件的安装方法类似
		\item[] 一直下一步稍加一些自定义(如安装路径)即可
		\item[] {\bf 注意:} 安装程序在某些情况下可能覆盖PATH环境变量,请在安装前注意备份PATH环境变量
	\end{itemize}
\end{description}

\section{\TeX{}Live [ Windows \& Linux ]}

\TeX{}是自由软件,有很多发行版本,就像Linux的Ubuntu、Fedora等等。
每个发行版本都是一套工具集合,包括plain\TeX{},\LaTeX{},pdf\TeX{},dvips等。
其中比较流行的是\TeX{}Live,也包含在CTAN的开源镜像中。

推荐通过下载ISO镜像文件的方式安装:
\begin{description}
	\item[官方说明]
	\url{http://www.tug.org/texlive/acquire-iso.html}
	\item[下载地址] 官方地址会自动跳转寻找"最近"镜像,还有几个较快的教育网镜像
	\begin{description}
		\item[官方地址]
		\url{http://mirror.ctan.org/systems/texlive/Images/texlive2016.iso}
		\item[清华镜像]
		\url{http://mirrors.tuna.tsinghua.edu.cn/CTAN/systems/texlive/Images/}
		\item[中科大镜像]
		\url{https://mirrors.ustc.edu.cn/CTAN/systems/texlive/Images/}
	\end{description}
	\item[安装方法] \hfill
	\begin{enumerate}
		\item 通过虚拟光驱挂载镜像也可以直接打开或解压缩不过会比较慢
		\item 双击运行光盘镜像或者运行脚本
		\item[] Windows 用户可以直接双击运行\textsl{install-tl.bat}
		\item[] Linux 用户可以在终端下执行命令\textsl{./install-tl}
		\item 按照提示下一步即可,安装大致耗时10$\sim$20分钟,受机器配置影响。
	\end{enumerate}
\end{description}

当然官方也提供了通过网络安装的方式,虽然通过可以通过镜像选择达到比较快的速度,
但是这里简便期间不再赘述,有兴趣的同学可以参考官方说明
\url{http://www.tug.org/texlive/acquire-netinstall.html}。

\section{Mac\TeX{} [ Mac ]}

Mac\TeX{}是基于\TeX{}Live为Mac系统设计的套件。

\begin{description}
	\item[官方网站]
	\url{http://tug.org/mactex/}
	\item[下载地址] 官方地址会自动跳转寻找"最近"镜像,还有几个较快的教育网镜像
	\begin{description}
		\item[官方地址]
		\url{http://mirror.ctan.org/systems/mac/mactex/MacTeX.pkg}
		\item[清华镜像]
		\url{http://mirrors.tuna.tsinghua.edu.cn/CTAN/systems/mac/mactex/}
		\item[中科大镜像]
		\url{https://mirrors.ustc.edu.cn/CTAN/systems/mac/mactex/}
	\end{description}
	\item[安装方法] 同一般软件安装,下一步即可
\end{description}

\section{兼容性}

本模板依赖v2.0及以上版本的ctex包,\TeX{}Live 2015及以上版本、C\TeX{}2.9.3 可以正常使用。
对于低版本的\LaTeX{}发行版,需要使用包管理器升级ctex宏包。

\section{安装字体 [ Linux ]}

该毕业设计论文要求使用Times New Roman、华文新魏和华文行楷这三种字体,在Linux系统上,这些字
是没有预装在系统里的,因此Linux用户需要手动安装字体才能正常使用本模板。本节将以Ubuntu系统
为例演示字体的安装。

首先需要获取字体文件,Windows系统默认包含了Times New Roman和华文行楷这两种字体,可以从
\verb|C:\Windows\Fonts\|文件夹下将字体文件拷贝出来(显示为\verb|STXingkai|和
\verb|Times New Roman|),当然,用户也可以从其他途径获取这两个字体文件。然后将字体文件
拷贝到Ubuntu的\verb|/usr/share/fonts|目录下,为了方便管理,可以在这些外部字体放在一个新
文件夹中:
\begin{lstlisting}[
	language={bash},
	caption={拷贝字体文件},
	label={copy-fonts},
	]
	sudo cp <your font files> /usr/share/fonts/msfonts/
\end{lstlisting}
然后将字体文件的权限设置为644:
\begin{lstlisting}[
	language={bash},
	caption={设置字体文件权限},
	label={set-fonts-permission},
	]
	sudo chmod 644 /usr/share/fonts/msfonts/*
\end{lstlisting}
接下来,进入到{\verb /usr/share/fonts/msfonts } 目录下,依次运行以下三个命令:
\begin{lstlisting}[
	language={bash},
	caption={安装字体},
	label={install-fonts},
	]
	sudo mkfontscale
	sudo mkfontdir
	sudo fc-cache -fv
\end{lstlisting}
当看到命令行输出
\begin{lstlisting}[
	language={bash},
	caption={正常输出结果},
	label={install-font-success},
	]
	fc-cache: succeeded
\end{lstlisting}
时,就完成了字体的安装。

\section{关于编辑器}

以上介绍了三款\LaTeX{}套装,涵盖了主流的三大平台,除了C\TeX{}自带了WinEdt,
其余两款均需要自己选择编辑器,理论上任何文本编辑器都是可以使用的,
如Windows上的 vscode,Linux/MacOS上的vim,emacs,
一方面要考虑对\LaTeX{}的支持,一方面还是自己的熟悉程度。

这里推荐一款大众化的编辑器\TeX{}maker,它是跨平台的,支持Windows、Linux和MacOS。

\begin{description}
	\item[官方网站]
	\url{http://www.xm1math.net/texmaker/}
	\item[下载地址]
	\url{http://www.xm1math.net/texmaker/download.html}
	\item[相关说明]
	\begin{itemize}
		\item 安装同一般软件的安装
		\item 配置Xe\LaTeX{}的编译,选择菜单栏“选项”->“配置\TeX{}Maker”,
		\item[] 在“\LaTeX{}”一栏填写
		\texttt{xelatex -interaction=nonstopmode\%.tex}
	\end{itemize}
\end{description}

\section{关于编译}

\LaTeX{}的文件是通过编译生成的,对于本模板和毕业设计论文而言,
需要经过代码\ref{code-compile}所示步骤(以sample-bachelor.tex为例):
\begin{lstlisting}[
	language={bash},
	caption={编译步骤},
	label={code-compile},
	]
	xelatex sample-bachelor.tex
	bibtex  sample-bachelor.aux
	xelatex sample-bachelor.tex
	xelatex sample-bachelor.tex
\end{lstlisting}
当然,我们在模板里也提供了编译的执行脚本。

\subsection{批处理 [ Windows only ]}

进入cmd(Win+R,然后输入cmd),cd到BUAAthesis对应目录,
如\verb|D:\BUAAthesis\|,然后运行\verb|msmake|即可。

\subsection{Makefile [ Windows(cygwin) / Linux / MacOS ]}
需要要你的命令行环境支持Make,cd到BUAAthesis相应目录,
目前支持如代码\ref{code-make}所示的功能:
\begin{lstlisting}[
	language={bash},
	caption={make 命令},
	label={code-make},
	]
	make bachelor # 编译本科生的\LaTeX{}(文件默认项,亦可直接输入make)
	make master # 编译研究生的\LaTeX{}文件
	make kaitireport # 编译本科生/研究生的开题报告/文献综述的\LaTeX{}文件
	make clean # 删除编译过程中生成的文件(除了pdf)
	make depclean # 删除编译过程中生成的文件(包括pdf)
\end{lstlisting}
